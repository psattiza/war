\documentclass[twocolumn,11pt]{article}

\usepackage{fontspec}
\usepackage{graphicx}
\usepackage{hyperref}

\title{\textbf{WAR: WAR with Auction Rounds} \\
{\large Report for \emph{Game Theory \& Networks} with \emph{Dr.~Dejun Yang}}
}
\author{
    Jack Rosenthal \\
    \texttt{jrosenth@mines.edu}
    \and Paul Sattizahn \\
    \texttt{psattiza@mines.edu}}
\date{2017-12-15}

\begin{document}
\maketitle

\section{Introduction}

\emph{War} is a card game played with two players and a single deck of cards.
The deck of cards is shuffled, and each player is dealt half of the deck. The
players repeatedly flip the top card of their deck, the player flipping the
higher valued card wins both cards and puts the cards in their \emph{wins
pile}. If the two players play the same card, they ``go to war''. In a war (the
resolution of a tie), both players discard the top three cards from their deck
to the pot, then play a fourth. The fourth card decides who wins the war. If
the fourth card is the same, the players go to war again until the war is
resolved. Once a player runs out of cards in their deck, they flip their wins
pile and continue to play. If they have no wins pile, the other player wins.

Studying War would be very boring from a game theoretical approach, as the game
involves no element of strategy. Instead, we propose a modified game: WAR -- a
recursive acronym standing for \textbf{WAR with Auction Rounds}. To distinguish
these two games, we will refer to the original game of War as \emph{Regular
War}, and our modified version as \emph{WAR} (as printed in all capital
letters).

WAR is a game played with $N$ players, each player starts with one suite of
cards (13 cards, labelled $2, 3, \ldots, 10, \mathrm{J}, \mathrm{Q},
\mathrm{K}, \mathrm{A}$). The players then repeatedly \emph{choose} a card from
their hand to play, the cards are then flipped simultaneously and the player
who plays the highest card takes all the cards for their wins pile. Similar to
Regular War, when there is a tie for the highest card, all the players who
played the highest card go to war. Unlike Regular War, during a war, each of
the players involved get to \emph{choose} which three cards to discard and
which card to battle with. This process is repeated until the war is resolved.
Finally, just like Regular War, when a player is out of cards in their hand,
they pick up their wins pile if they have one, otherwise, they are out of the
game.

To assist in your understanding of the game, the authors have produced a video
demonstration of the game. This video is available online at
\url{https://www.youtube.com/watch?v=KzWVCbNsjwc}.

In this paper, we will define a model for the game, and discuss various
strategies (and their relation in terms of effectiveness to the other
strategies) we discovered.

\end{document}
